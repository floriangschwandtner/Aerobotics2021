\documentclass[a4paper,ngerman,11pt,chapterprefix=false,oneside,openright]{scrreprt}
\usepackage[utf8]{inputenc}
\usepackage[ngerman]{babel}
\usepackage[headings]{fullpage}
\newcommand\tab[1][1cm]{\hspace*{#1}}

%Grafikeinbindung
\usepackage[pdftex]{graphicx,xcolor}
\graphicspath{{./src/pics/}}
\usepackage{pgfplots}
\pgfplotsset{compat=1.15}
\usepackage{tikz}
\usetikzlibrary{plotmarks,shapes,arrows.meta,decorations.markings,patterns,calc}
% Zum flexiblen Einbinden von Grafiken in .svg Format
\usepackage{svg}
\svgsetup{inkscape=false}

%Units
\usepackage{siunitx}
\sisetup{locale = DE} 

%Kopfzeile
\usepackage{geometry}
\geometry{verbose,a4paper,tmargin=3.5cm,bmargin=2.5cm,lmargin=2.6cm,rmargin=2.6cm,headheight=40pt,headsep=1cm}
\usepackage[headsepline= 0.4pt]{scrlayer-scrpage}
\pagestyle{scrheadings}
%\clearscrheadfoot % deprecated
\clearpairofpagestyles
\ohead{  \normalfont \sffamily \bfseries \small Universität Stuttgart \\[2pt] Institut für Flugmechanik und Flugregelung  \\[8pt] \normalfont\sffamily\footnotesize{Prof. Dr.-Ing. Walter Fichter}\\[-37pt] }
\lohead{ \includegraphics[scale=0.4, trim= -0.0cm 0.05cm 0cm 0cm] {./src/pics/logo/ifrlogo.pdf} \vspace{0.05cm}}
\cfoot[\pagemark]{\pagemark}

%Mathematikfunktionen
\usepackage{amsmath,amsfonts,amssymb}
\usepackage{mathtools}

% ToDo-Notes
\usepackage{todonotes}

% References
\usepackage[hidelinks]{hyperref}
\usepackage{cleveref}

% % Tabellen- und Abbildungspakete 
% Multirow in Tabellen:
% Ermöglicht das Verbinden von Zeilen in Tabellen
\usepackage{multirow}
% Für schöne Tabellen, bietet dicke/dünne Linien an
% mehr Tabelleneinstellungen ( \toprule / \midrule / \bottomrule)
\usepackage{booktabs}
\renewcommand{\arraystretch}{1.3}		% Erhöht den Zeilenabstand von Tabellen leicht (Ästhetik)
\setlength{\tabcolsep}{5pt}
% Beide Pakete werden für die Ausrichtung der Tabellenspalten benötigt
\usepackage{array}
% Für lange Tabellen
\usepackage{longtable}
\usepackage{tabularx}				% mehr Tabelleneinstellungen (z.B. Tabellen mit benutzerdefinierter Breite)
%\usepackage{tikz}					% Erstellen von Vektorgrafiken in LaTeX
\pgfplotsset{compat=1.15}			% Neuste Version des Pakets aufrufen 
\pgfplotsset{plot coordinates/math parser=false} % (benötigt von matlab2tikz)
\newlength\fwidth					% Breite einer Matlab-Figure bestimmen mit \setlength\fwidth{}
\newlength\fheight					% Höhe einer Matlab-Figure bestimmen mit \setlength\fheight{}
% Darstellung für Caption
\usepackage[font=small,labelfont=bf,labelsep=endash,format=plain]{caption}

\makeindex
  
%==================================================================
 %==================================================================
 % Metadaten der Arbeit
 %
 % Titel
 \newcommand{\ifrtitleOne}{Systemidentifikation: }
 \newcommand{\ifrtitleTwo}{Schätzung der Parameter flugmechanischer Modelle aus Flugmessdaten}
 % Author
  \newcommand{\ifrauthor}{Calvin Ebert\\Adam Ghribi\\Florian Gschwandtner\\Fabrizio Turco}
 %
 %=================================================================== 
 %==================================================================
  

%Beispieltext
\usepackage{blindtext}

\begin{document}
%Titlepage
%********************************
% Titelseite
%********************************
\section*{
\vspace{4.5cm}
{\LARGE \\ 
Aerobotics-Seminar  \\
Moonshot-Aufgabe   \vspace{1.5cm}\\ 
\ifrtitleOne \\
\huge \ifrtitleTwo \\
\vspace{4.3cm} 
} 
{\normalsize
Autoren: Gruppe 02\\ 
\ifrauthor \vspace{1cm}\\
Datum: 06.08.2021
}
}


%Table of contents
\tableofcontents
%\pagenumbering{arabic} 

%Chapters
\chapter{Einleitung}

In der vorliegenden Arbeit geht es darum, aus vorliegenden Flugmessdaten des e-Genius 1:3-Modells die Parameter 
linearisierter Modelle der Längs- und Seitenbewegung zu identifizieren. Dazu werden Verfahren im Frequenz- und im Zeitbereich 
verwendet, um die Ergebnisse vergleichen zu können.

- wichtige Rolle in der Regelungstechnik: liefert Modell der Regelstrecke

	o gegeben: 	$ x(t) $, $ u(t) $ aus Flugmessdaten, Struktur der lin. Modelle (Längs- und Seitenbewegung)
	o gesucht: 	Parameter der lin. Modelle

- 
\chapter{Einleitung}

Im folgenden Abschnitt wird das der durchgeführten Systemidentifikation zugrundeliegende Modell beschrieben. Es handelt sich 
dabei um die bekannten linearisierten Modelle der Längs- und Seitenbewegung \cite{Fichter2009} mit den zu bestimmenden 
Beiwerten.

\section{Längsbewegung}

\chapter{Vorbereitung der Daten}
Aus den Flugversuchen des e-Genius 1:3 liegt eine Fülle von Messdaten in verschiedenen .csv-Dateien vor. Im folgenden Kapitel 
wird erklärt, wie diese vorbereitet werden, um sie für die anschließende Systemidentifikation zu verwenden.



\section{Ermittlung der relevanten Signale} % Fabrizio
%\todo[inline]{Abschnitt "Ermittlung der relevanten Signale" schreiben}
Für die weitere Verarbeitung ist es zunächst nötig, aus den gegebenen Messdaten die relevanten Signalverläufe auszuwählen 
bzw. zu berechnen. Die meisten Zustandsgrößen können direkt aus den Messdaten verwendet werden, einzig der Bahnwinkel $ 
\gamma $ muss explizit berechnet werden. Die Bestimmung über die Beziehung $ \gamma=\theta-\alpha $ liefert dabei aufgrund 
eines unplausiblen Verlaufs des Nickwinkels $ \theta $ kein sinnvolles Ergebnis.\footnote{Berechnet man den Bahnwinkel auf 
diese Weise, nimmt er genau wie der Nickwinkel nie Werte kleiner 0 an. Lediglich der Anstellwinkel bewegt sich sowohl im 
positiven als auch im negativen Bereich.} Stattdessen wird der Bahnwinkel mit \cref{eq:gammaCalculation} über die 
geodätischen Geschwindigkeiten in $ z $- und $ x $-Richtung berechnet.

\begin{equation}
	\gamma = \arcsin{\left( \frac{V_z}{V_x} \right)}
	\label{eq:gammaCalculation}
\end{equation}

\cref{tab:messgroessen} zeigt eine Übersicht über die verwendeten Messsignale aus den Originaldateien. Zu beachten ist, dass 
in den Originaldaten für sämtliche Steueruder jeweils ein eigenes Signal für das linke sowie das rechte Ruder vorliegt (im 
Fall des Querrruders sogar zwei Signale je Seite). Im hier betrachteten Bereich sind diese Signale jedoch zu jedem Zeitpunkt 
gleich, weshalb die Verwendung eines Verlaufs ausreicht.

\begin{table}[h!]
	\centering
	\caption{Übersicht über die verwendeten Messgrößen}
	\label{tab:messgroessen}
	\begin{tabular}{@{} cll @{}} 
		\toprule
		\multicolumn{3}{c}{Längsbewegung}\\
		\midrule
		Größe 	&Messsignal &Datei\\
		\midrule
		$ \alpha $	&Alpha &vectoflow\_airdata\\
		$ q $	&pitchspeed &vehicle\_attitude\\
		$ V_A $	&VMag &vectoflow\_airdata\\
		$ \gamma $ &vx, vz &vehicle\_local\_position\\
		$ \eta $ &elevator\_l &actuator\_controls\\
		$ \delta_F $ &thrust &actuator\_controls\\
		\bottomrule
	\end{tabular}
	\begin{tabular}{@{} cll @{}} 
		\toprule
		\multicolumn{3}{c}{Seitenbewegung}\\
		\midrule
		Größe 	&Messsignal &Datei\\
		\midrule
		$ r $	&yawspeed &vehicle\_attitude\\
		$ \beta $	&Beta &vectoflow\_airdata\\
		$ p $	&rollspeed &vehicle\_attitude\\
		$ \phi $ &Phi &vectoflow\_airdata\\
		$ \xi $ &aileron\_inner\_l &actuator\_controls\\
		$ \zeta $ &rudder &actuator\_controls\\
		\bottomrule
	\end{tabular}
\end{table}

Die vorliegenden Messdaten umfassen einen großen Zeitbereich von Start bis Landung. Für die Systemidentifikation wurde nur 
der Abschnitt berücksichtigt, in dem das Flugzeug Platzrunden fliegt. Dies entspricht im originalen Datensatz in etwa der 
Zeit zwischen $ \SI{910}{\second} $ und $ \SI{2225}{\second} $. \cref{fig:flug} zeigt den Flug und den entsprechenden 
Ausschnitt.

\begin{figure}[h!]
	\centering
	\includegraphics[width=0.9\linewidth]{flug.png}
	\caption{Visualisierung der Flugbahn}
	\label{fig:flug}
\end{figure}

\section{Trimmpunkt} % Fabrizio
In \cref{fig:trimmpunkte} ist beispielhaft der zeitliche Verlauf der Anströmgeschwindigkeit dargestellt. Es zeigen sich 
insgesamt acht stationären Bereiche, die als Trimmpunkt für eine Modellierung dienen könnten. In den nachfolgenden 
Systemidentifikationen soll jeweils der erste Punkt (TP 1) als Grundlage dienen. Dazu werden alle Zustands- und Steuergrößen 
über den Bereich TP 1 gemittelt und diese Mittelwerte als Trimmwerte $ x_0 $ und $ u_0 $ verwendet. Damit lassen sich die 
Abweichungen vom Trimmpunkt berechnen:
\begin{equation}
	\begin{split}
		\Delta x(t) &= x(t)-x_0\\
		\Delta u(t) &= u(t)-u_0
	\end{split}
\end{equation}

Eine Ausnahme bilden hier die Drehraten $ p $, $ q $ und $ r $, bei denen keine Differenz zum Trimmwert gebildet werden muss.


\begin{figure}[h!]
	\centering
	\includegraphics[width=0.9\linewidth]{trimmpunkte.png}
	\caption{zeitlicher Verlauf der Anströmgeschwindigkeit mit den stationären Bereichen}
	\label{fig:trimmpunkte}
\end{figure}




\section{Interpolation} %Florian
Die Rohdaten werden von mehreren Sensoren mit unterschiedlichen Abtastraten geliefert. Da die Identifikationsalgorithmen die 
Werte zu diskreten Zeitpunkten benötigen, ist es notwendig, einen einheitlichen Zeitvektor mit zugehörigen Eingangs- und 
Zustandsvektoren zu generieren. Außerdem wird eine konstante Schrittweite gefordert.

Die Abtastrate dieses Zeitvektors ist wichtig, da die Dimension des Optimierungsproblems und damit der 
Rechenaufwand mit feinerer Diskretisierung steigt (im Zeitbereich). In dieser Arbeit wird der Zeitvektor des 


\begin{figure}[h!]
	\centering
	\includegraphics[width=0.7\linewidth]{beispielInterpolation.png}
	\caption{Beispiel Interpolation}
	\label{fig:interpBsp}
\end{figure}

%fkt interp1

\section{Filterung} %Florian

Verrauschte Messdaten stellen für die Systemidentifikation eine Herausforderung dar. Numerische Ableitungen aus verrauschten Daten liefern in vielen Fällen keine sinnvolle Aussage. Neben aufwändigeren Ableitungsregeln bietet sich eine vorangehende Filterung der Daten an.

Das Vorwärts-Rückwärtsfilter bietet den Vorteil, dass keine Phasenverschiebung 
auftritt. Gerade wenn nur einzelne Signalteile gefiltert werden, beispielsweise 
nur der Eingang, ist diese Eigenschaft unerlässlich. Der Nachteil ist, dass das 
Filter nicht in Echtzeit verwendet werden kann, da immer die vollständige 
Datenreihe vorliegen muss. Für eine Systemidentifikation ist dies jedoch keine 
praktische Einschränkung.

In \cref{fig:filterBsp} sind die Auswirkungen einer reinen Vorwärts- und einer Vorwärts-Rückwärtsfilterung auf die Phase 
gut zu sehen.

\begin{figure}[h!]
	\centering
	\includegraphics[width=0.7\linewidth]{beispielFilterung.png}
	\caption{Beispiel Filterung}
	\label{fig:filterBsp}
\end{figure}


\subsection{Ablauf}

Für das Filter wird eine Übertragungsfunktion $f(s)$ auf die Messdaten vorwärts 
angewandt, die Messdaten umgekehrt und die selbe Übertragungsfunktion noch 
einmal verwendet. In Matlab ist dies in der Funktion \textit{filtfilt()} 
bereits implementiert. 

\subsection{Wahl der Filterübertragungsfunktion}

Es wurde ein quadriertes PT2-Glied gewählt, da so die Eckfrequenz direkt eingestellt werden kann. Die Eckfrequenz wurde zu $ 
\SI{20}{\hertz} $ gewählt, damit das Rauschen unterdrückt wird, aber keine Information verloren geht.

Mit

\begin{equation}
	\omega_{filt} = 2 \cdot \pi \cdot f_{eck} = 2 \cdot \pi \cdot \SI{20}{\hertz}
\end{equation}

%\todo{gewählte Eckfrequenz nennen}

und der Dämpfung

\begin{equation}
	\zeta_{filt} = \frac{1}{\sqrt{2}}
\end{equation}

ergibt sich die Übertragungsfunktion zu:

\begin{equation} \label{eq:filter}
	f(s) = \left(\frac{\omega_{filt}^2}{s^2+2 \cdot \zeta_{filt} \cdot \omega_{filt} +\omega_{filt}^2}\right)^2
\end{equation}

Im Bodediagramm, \cref{fig:bodeplot}, sind Amplituden- und Frequenzgang der Filterfunktion zu sehen.

\begin{figure}[h!]
	\centering
	\includegraphics[width=0.7\linewidth]{bodeplot.png}
	\caption{Bodediagramm der Filterfunktion aus \cref{eq:filter}}
	\label{fig:bodeplot}
\end{figure}
\chapter{Systemidentifikation im Zeitbereich}

\section{Methode der kleinsten Fehlerquadrat}
Die Methode der Kleinste-Quadrate-Schätzung (LSQ) wird in diesem Kapitel vorgestellt.
Eine der Kernaufgaben vieler Ingenieure einen belastbaren Zusammenhang zwischen gemessenen Werten zu finden. Diese Zusammenhänge können  Näherungsweise durch folgendes lineares Modell beschrieben werden:  
\begin{equation}
    z = H\cdot x+v 
\end{equation}
mit 
\[z = [z_{1} \, z_{2} \, ...\,  z_{m} ]^T =\, m\times 1 \; \text{Messungsvektor} \]
\[ H= \; m\times n \;\text{bekannte Matrix } \]
\[ v = \;\text{unbekannter Rauschterm } \]
\[ x= \; n\times 1 \;\text{unbekannter Parametervektor } \]
 \noindent LSQ kann hilfreich werden, um zum Beispiel die Schätzung von Modellparametern bei Systemidentifikation durchzuführen.  
Gesucht ist ein Schätztwert $\hat{x}$ für Parametervektor. Der Restfehlervektor $e$ lautet:
\begin{equation}
    e = z - H\cdot \hat{x} \\
\end{equation}
Die Idee ist ein Wert von $\hat{x}$ zu finden, das die 2-Norm des Restfehlervektors minimieren soll. Das quadratische Zielfunktional $J$ ist gegeben durch: 
\begin{align}
   \begin{split}
     J = 1/2 \cdot e^{T} \cdot e \\
     = 1/2 \cdot {(z- H\hat{x})}^{T}\cdot(z- H\hat{x}) \\
     = 1/2 \cdot (z^{T} -{\hat{x}}^{T}H^{T})\cdot(z- H\hat{x}) \\
     = 1/2 \cdot z^{T}z - z^{T}H\hat{x} + 1/2\cdot {\hat{x}}^{T}A\hat{x}  \\
   \end{split}
\end{align}
mit $A = H^{T}H$. Die Lösung des Minimierungsproblems lautet dann:
\begin{equation}
    \hat{x}= {(H^{T} H)}^{-1} \cdot H^{T} z 
\end{equation}
\subsection{Bestimmung von Aerodynamischen Parametern anhand LSQ} 
Die Zustandsraumdarstellung \eqref{eq:laengsbewegung} liefert folgende Gleichungen: 
\[\Delta\dot \alpha-q =  \frac{Z_{\alpha}\Delta\alpha}{V_0} + \frac{Z_{\nu}\Delta V_{A}}{V_0} + \frac{Z_{\eta}\Delta\eta}{V_0} - \frac{X_{\delta F}\sin{(\alpha_0 + i_F)}\cdot\Delta\delta_F}{V_0}\] 
\[ \dot q = M_{\alpha}\Delta\alpha + M_{q}q + M_{\nu}\Delta V_A + M_{\eta}\Delta\eta + M{\delta F}\Delta\delta_F\]
\[\Delta\dot V_A = X_{\alpha}\Delta\alpha +  X_{\nu}\Delta V_A + X_{\eta}\Delta\eta + X_{\delta F}\cos{(\alpha_0 + i_F)}\cdot\Delta\delta_ \]
\[\Delta \dot \gamma = \frac{- Z_{\alpha}\Delta\alpha}{V_0} + \frac{- Z_{\nu}\Delta V_{A}}{V_0} + \frac{- Z_{\eta}\Delta\eta}{V_0} + \frac{X_{\delta F}\sin{(\alpha_0 + i_F)}\cdot\Delta\delta_F}{V_0}\]
Welche zu dem folgenden linearen Gleichungssystem umgeformt werden können: 
\begin{equation}
    z_{L}= H_{L}\cdot x + v
\end{equation}
Sodass:
\setcounter{MaxMatrixCols}{15}
\[z_{L} = (\Delta\dot \alpha-q \;\; \dot q \;\; \Delta\dot V_A)^T\] \\
\[ H_{L} = \begin{pmatrix}
0&0&0& \frac{-\sin{(\alpha_0 + i_F)}\cdot\Delta\delta_F}{V_0} & \frac{\Delta\alpha}{V_0}& \frac{\Delta V_A}{V_0} & \frac{\Delta\eta}{V_0} &0&0&0&0&0   \\
0&0&0&0&0&0&0 &\Delta\alpha & q & \Delta V_A & \Delta\eta & \Delta\delta_F \\
\Delta\alpha &  \Delta V_A & \Delta\eta & \cos{(\alpha_0 + i_F)}\cdot\Delta\delta_F &0&0&0&0&0&0&0&0 
\end{pmatrix}\] \\
\[x = (X_{\alpha}\; 
X_{\nu}\;
X_{\eta}\;
X_{\delta_F}\; 
Z_{\alpha}\; 
Z_{\nu}\;
Z_{\eta}\;
M_{\alpha}\;
M_{q}\;
M_{\nu}\;
M_{\eta}\;
M_{\delta_F}\;)^T
 \]
 \section{Ergebnisse}
 
 Anschließend werden die Ergebnisse in diesem Kapitel kurz vorgestellt.\par
 \noindent Nach der Schätzung der Einträge der Matrizen $A$ und $B$, wird das Anfangswertproblem \eqref{eq:laengsbewegung} auf das gesamte Zeitintervall gelöst. Bei allen Simulationen sind die Filterparamtern von Kapitel 3.4.2 gewählt.\par
 \noindent In Abbildung \ref{fig:Ergebnisse_zmlsq} wird die Lösung Anhand ein Matrix-LSQ-Verfahren dargestellt. Die Ergebnisse zeigen eine relativ präzise Approximation. Das verfahren ist sehr schnell und einfach zu implementieren in vergleich zur LSQ-Methode.  Allerdings steht es aus flugmechanischer Sicht Abweichungen in System- und Eingangsmatrix. Die Matrizen $A$ und $B$ lauten: 
 \[ A = \begin{pmatrix}
 -0.1940 & 0.005 & -0.0002 & -0.0540 \\
 -38.9899 & -12.8888 & 0.3632 & -4.2686 \\
 -6.1506 & -0.1501 & -0.0589 & -5.0107 \\
 0.4433 & 0.0427 & 0.0023 & 0.0818
 \end{pmatrix}
 \] 
 \[ B = \begin{pmatrix}
 -0.0267 & -0.0164 \\ 
 -26.1377 & -12.9651 \\
 -0.7414 & 4.6092 \\
 0.0477 & -0.1328
 \end{pmatrix}
 \]
 \noindent Die Abbildung \ref{fig:Ergebnisse_zlsq} stellt die Lösung Anhand das LSQ-Verfahren. Die Methode liefert keine sonderlich guten Ergebnisse. 
 \begin{figure}[h!]
	\centering
	\includegraphics[trim=100 0 100 0,clip,width=1\linewidth]{LS.png}
	\caption{Ergebnisse Anhand Matrix-LSQ }
    \label{fig:Ergebnisse_zmlsq}
\end{figure}

 \begin{figure}[h!]
	\centering
	\includegraphics[trim=100 0 100 0,clip,width=1\linewidth]{LS_LSQ.png}
	\caption{Ergebnisse Anhand LSQ}
     \label{fig:Ergebnisse_zlsq} 
\end{figure}

\chapter{Systemidentifikation im Frequenzbereich}

Neben der Systemidentifikation im Zeitbereich ist es möglich die Parameterschätzung auch im Frequenzbereich durchzuführen. 
Die Analyse im Frequenzbereich bringt einige Vorteile mit sich, die im weiteren Verlauf dieser Dokumentation genauer 
beleuchtet werden. Die Grundlage der Frequenzanalyse ist die Fouriertransformation der gemessenen Daten vom Zeitbereich in 
den Frequenzbereich. Ausgehend von den transformierten Messwerten wird im Folgenden genauer auf die Idee der 
Output-Error-Methode und auf den Algorithmus zur Lösung des Schätzproblems eingegangen. Zum Schluss werden die Ergebnisse 
bewertet und der Blick auf mögliche Verbesserungen und andere Methoden gerichtet.


\section{Fouriertransformation}  

\begin{align}
	j\omega_{k}\mathbf{\tilde{x}}(k)  &= \mathbf{A\tilde{x}}(k) + \mathbf{B\tilde{u}}(k) \nonumber \\
 	\mathbf{\tilde{y}}(k)             &= \mathbf{C\tilde{x}}(k) + \mathbf{D\tilde{u}}(k) = \mathbf{G}(k,\mathbf{\theta})\mathbf{\tilde{u}}(k) \nonumber \\
 	\text{mit }\mathbf{G}(k,\mathbf{\theta}) &= \mathbf{C}(j\omega\mathbf{I}-\mathbf{A})^{-1}\mathbf{B}+\mathbf{D}
	\label{eq:ZRD_Frequenzbereich}
\end{align}

\missingfigure{Bild von Fourier-Trafo}


\section{Output-Error-Methode}

Die Idee der Output-Error-Methode (OEM) liegt in der Art der Fehlerbetrachtung. Es wird dabei der Fehler, der durch die 
mathematische Modellbildung entsteht, vernachlässigt. Das dynamische System wird als deterministisch angenommen. 
Unsicherheiten entstehen nur durch fehlerhafte und verrauschte Messungen. Ausgangspunkt der OEM ist das transformierte 
Zustandsraummodell (\ref{eq:ZRD_Frequenzbereich}). Im vorliegenden Fall entspricht der Ausgangsvektor $\tilde{y}$ dem 
Zustandsvektor $\tilde{x}$. Die Matrizen $\mathbf{C}$ und $\mathbf{D}$ nehmen also folgende Gestalt an:

\begin{align}
	\mathbf{C} &= \mathbf{I} \nonumber \\
 	\mathbf{D} &= \mathbf{0}          
	\label{eq:CD}
\end{align}

Da der Zustandsvektor gleichzeitig der gemessene Zustand ist, kann der Output-Error folgendermaßen formuliert werden:

\begin{equation}
    \mathbf{\tilde{\nu}}(k,\mathbf{\theta}) = \mathbf{\tilde{z}}(k)-\mathbf{\tilde{y}}(k,\mathbf{\theta}) = \mathbf{\tilde{z}}(k)-\mathbf{G}(k,\mathbf{\theta})\mathbf{\tilde{u}}(k)  
	\label{eq:Output_Error}
\end{equation}

$\mathbf{\tilde{z}}$ entspricht dabei dem gemessenen Zustand, der durch Messrauschen aus dem übertragenen Zustand 
$\mathbf{\tilde{y}}$ entsteht. Der übertragene Zustand kann aus den Steuergrößen $\mathbf{\tilde{u}}$ durch Multiplikation 
mit der Übertragungsmatrix $\mathbf{G}(\mathbf{\theta)}$ berechnet werden. Die Übertagungsmatrix ist, wie in 
\cref{eq:ZRD_Frequenzbereich} gezeigt, abhängig vom zugrunde liegenden Modell und von den zu schätzenden Parametern 
$\mathbf{\theta}$. Diese sind die Einträge der Matrizen des Zustandsraummodells. Das Ziel der Methode ist nun den Ausgangsvektor, der aus dem Modell und seinen Parametern folgt, dem gemessenen Zustand möglichst gut anzunähern. Mathematisch bedeutet das, dass das Minimum einer Kostenfunktion gefunden werden soll. Die Kostenfunktion der OEM ist nach Klein und Morelli \cite{Klein2006} die negative Log-likelihood Funktion mithilfe des Ansatzes des Ausgangsfehlers (Gl. \ref{eq:Output_Error}):

 \begin{equation}
    J(\mathbf{\theta})=N \sum\limits_{k=0}^{N-1}\mathbf{\tilde{\nu}}^H(k,\mathbf{\theta})\mathbf{S_{\nu\nu}^{-1}}\mathbf{\tilde{\nu}}(k,\mathbf{\theta})+Nln|\mathbf{S_{\nu\nu}}|
	\label{eq:Kostenfunktion}
\end{equation}  

\todo{Svv kurz in ein zwei Sätzen erklären}
Wie bereits beschrieben geht es nun darum das Minimum dieser Kostenfunktion zu finden. Wir suchen also die Nullstelle ihrer Ableitung. Einer der wichtigsten Algorithmen zur Bestimmung von Nullstellen sowohl bei Eingrößenproblemen, als auch bei Mehrgrößenproblemen ist der iterative Newton Raphson Algorithmus.

\section{Ergebnisse}
\chapter{Zusammenfassung und Ausblick}\label{chapter:Zusammenfassung}
Abschließend lässt sich sagen, dass die Datenvorbereitung ein sehr wichtiger und auch zeitintensiver Teil der gesamten Arbeit 
war. Die richtigen Messsignale aus der Fülle an Originaldaten zu finden und auf Plausibilität zu prüfen legt den Grundstein 
für alle weiteren Schritte. Schwierig ist außerdem die Auswahl eines geeigneten Zeitbereichs zur Identifikation. 

Das LSQ-Verfahren im Zeitbereich lieferte keine sonderlich guten Ergebnisse. Deutlich bessere Approximationen der Messdaten 
ließen sich mit dem Matrix-LSQ-Verfahren erreichen. Dieses stellte sich als sehr schnell und vergleichsweise einfach in der 
Implementierung heraus. Aus flugmechanischer Sicht sind allerdings Einbußen in den bestimmten Parametern hinzunehmen, da sich 
keine Vorgaben bzgl. der bereits bekannten Einträge in der System- und Steuermatrix treffen lassen.

Es zeigte sich, dass der Flug in den Platzrunden weniger geeignet für eine Identifikatition im Frequenzbereich ist. Der 
Verlauf des Steuersignals ist hier von großer Relevanz, bestimmte Manöver, wie ein \textit{Frequency Sweep}, sollten hier zu 
deutlich besseren Ergebissen führen. Nichtsdestotrotz konnten bestimmte Zustandsgrößen angenähert werden. Der 
Implementierungsaufwand für die \textit{Output Error}-Methode ist deutlich höher verglichen mit den Zeitbereich-Verfahren; 
die symbolische Berechnung der sehr länglichen Ableitungen in Matlab ermöglichte überhaupt erst eine sinnvolle Anwendung.\\

An dieser Stelle sei erneut erwähnt, dass in der vorliegenden Arbeit lediglich der Platzrundenflug berücksichtigt wurde. In 
den restlichen Flugdaten steckt aber womöglich ebenso Potenzial zur Bestimmung des flugmechanischen Modells, vor allem, wenn 
sich noch passende Manöver (z.B. Anregung der Eigenschwingung des Flugzeugs) finden lassen. Als nächster Schritt bietet sich 
also eine genaue Untersuchung des bisher vernachlässigten Flugs an.

Möglichkeiten zur Verfeinerung und Optimierung der Identifikation selbst gibt es viele. An erster Stelle bietet sich eine 
Normalisierung der Signale an, d.h. eine Skalierung mit den Maximalwerten eines jeden Verlaufs, sodass die Werte danach in 
der gleichen Größenordnung liegen. Dies ist sonst nicht der Fall, die Geschwindigkeit erreicht weit höhere Absolutwerte als 
alle anderen Zustände, was in den verwendeten Algorithmen einer stärkeren Gewichtung des Geschwindigkeitssignalls entspricht. 
Eine Gewichtung 
ließe sich aber außerdem bewusst vornehmen und damit der Fokus speziell auf wichtige Größen legen. Weiterhin könnten die 
zugrundeliegenden Modelle vereinfacht werden. In der Längsbewegung würde das bedeuten, zwei reduzierte Modelle für die 
Anstellwinkelschwingung einerseits und die Phygoide andererseits zu erstellen und die Identifikationen getrennt durchzuführen.

Weit aufwändiger, aber auch sehr vielversprechend, ist eine Identifikation im Zeitbereich mit einem nichtlinearen 
Modell\todo{Infos ergänzen}.

Im Frequenzbereich wurde mit der \textit{Output Error}-Methode nur der Fehler im gemessenen Signal berücksichtigt. Alternativ 
wäre hier eine Betrachtung des Modellfehlers bei gleichzeitiger Vernachlässigung des Messfehlers möglich, wie es die 
\textit{Equation Error}-Methode vorsieht. Zu guter Letzt lassen sich diese beiden Methoden auch kombinieren.

%- Datenvorbereitung sehr wichtig und großer Teil der Arbeit\\
%- Matrix-LSQ sehr schnell und einfach zu implementieren, ABER flugmechanisch nur bedingt sinnvoll\\
%- Frequenzbereich nur sinnvoll bei passendem Eingangssignal\\
%\\
%- generell: Datennormalisierung (Skalierung mit Maximalwerten => Absolutwerte in gleicher Größenordnung
%) bzw. -gewichtung\\
%- Zeitbereich: nichtlineare Modellbildung\\
%- Frequenzbereich: Equation Error-Methode, kombinierte Methode (Output Error + Equation Error)
%- Vereinfachung des Modells

%Bibliography
\addcontentsline{toc}{chapter}{Literaturverzeichnis}
\bibliographystyle{abbrvdin}
\bibliography{./src/bibfile}

% Anhang
\appendix
\chapter{Test des Algorithmus im Frequenzbereich}\label{appendix:test_algorithmus}

Zum Test der \textit{Output Error}-Methode mit dem \textit{Newton-Raphson}-Algorithmus wurde ein sinusförmiges Signal 
mit einer bestimmten Frequenz in der Höhenruderstuerung sowie ein Sinus-Signal im Anstellwinkel erstellt und der Algorithmus 
verwendet, um die entsprechende Über-tragung zu identifizieren. Wie sich in 
\cref{fig:testdaten1} erkennen lässt, wird die 
ursprüngliche Frequenz gut angenähert. Der Anstellwinkelverlauf wird vom Algorithmus mit einem leichten Phasenversatz noch 
gut getroffen (vgl. \cref{fig:testdaten2}).\par
Die anderen Zustände wurden in diesem Test nicht betrachtet, der Schubdrosselgrad wurde konstant gehalten.

\begin{figure}[!h]
	\centering
	\includegraphics[width=0.6\linewidth]{src/pics/Testdaten1}
	\caption{Betragskennlinie des Originalsignals sowie der Näherung}
	\label{fig:testdaten1}
\end{figure}

\begin{figure}[!h]
	\centering
	\includegraphics[width=0.6\linewidth]{src/pics/Testdaten2}
	\caption{Zeitverläufe des Originalsignals sowie der Näherung und der 
	Steuerungen}
	\label{fig:testdaten2}
\end{figure}


\end{document}