\chapter{Systemidentifikation im Zeitbereich}
\todo[inline]{kurze Einleitung schreiben}

\section{\textit{Least Squares}-Methode}
In diesem Kapitel wird die Methode der kleinsten Fehlerquadrate (\textit{Least Squares}, LSQ) vorgestellt.
Eine der Kernaufgaben vieler Ingenieure besteht darin, einen belastbaren Zusammenhang zwischen gemessenen Werten zu finden. 
Diese Zusammenhänge können näherungsweise durch folgendes lineares Modell beschrieben werden:  
\begin{equation}
    z = H\cdot x+v 
\end{equation}
mit 
\begin{align}
	\begin{split}
		\text{Messvektor: } &\dim{(z)} = m\times 1\\
		\text{bekannte Matrix: } &\dim{(H)} = m\times n\\
		\text{unbekannter Rauschterm: } &\dim{(v)} = m\times1\\
		\text{unbekannter Parametervektor: } &\dim{(x)} = n\times 1
		\nonumber
	\end{split}
\end{align}

Die LSQ-Methode kann beispielsweise verwendet werden, um die Schätzung von Modellparametern im Rahmen der 
Systemidentifikation durchzuführen. Gesucht ist dabei ein Schätzwert $\hat{x}$ für den Parametervektor. Der Restfehlervektor 
$e$ lautet:
\begin{equation}
    e = z - H\cdot \hat{x} \\
\end{equation}

Die Idee besteht darin, einen Wert von $\hat{x}$ zu finden, der die Norm des quadrierten Restfehlervektors minimiert. Das 
quadratische Zielfunktional $J$ ist gegeben durch: 
\begin{align}
   \begin{split}
     J &= \frac{1}{2} \cdot e^{T} \cdot e \\
     &= \frac{1}{2} \cdot {(z- H\hat{x})}^{T}\cdot(z- H\hat{x}) \\
     &= \frac{1}{2} \cdot (z^{T} -{\hat{x}}^{T}H^{T})\cdot(z- H\hat{x}) \\
     &= \frac{1}{2} \cdot z^{T}z - z^{T}H\hat{x} + \frac{1}{2}\cdot {\hat{x}}^{T}A\hat{x}  \\
   \end{split}
\end{align}
mit $A = H^{T}H$. Die Lösung des Minimierungsproblems lautet dann:
\begin{equation}
    \hat{x}= {(H^{T} H)}^{-1} \cdot H^{T} z 
\end{equation}


\subsection{Schätzung der Parameter mit dem LSQ-Verfahren} 

Die Zustandsraumdarstellung \eqref{eq:laengsbewegung} liefert folgende Gleichungen: 

\begin{align}
	\Delta\dot \alpha-q &=  \frac{Z_{\alpha}\Delta\alpha}{V_0} + \frac{Z_{\nu}\Delta V_{A}}{V_0} + 
	\frac{Z_{\eta}\Delta\eta}{V_0} - \frac{X_{\delta_F}\sin{(\alpha_0)}\cdot\Delta\delta_F}{V_0}\\
	\dot q &= M_{\alpha}\Delta\alpha + M_q q + M_{\nu}\Delta V_A + M_{\eta}\Delta\eta + M{\delta_F}\Delta\delta_F\\
	\Delta\dot V_A &= X_{\alpha}\Delta\alpha +  X_{\nu}\Delta V_A + X_{\eta}\Delta\eta + 
	X_{\delta_F}\cos{(\alpha_0)}\cdot\Delta\delta_F \\
	\Delta \dot \gamma &= \frac{- Z_{\alpha}\Delta\alpha}{V_0} + \frac{- Z_{\nu}\Delta V_{A}}{V_0} + 
	\frac{-Z_{\eta}\Delta\eta}{V_0} + \frac{X_{\delta F}\sin{(\alpha_0)}\cdot\Delta\delta_F}{V_0}
\end{align}
	
Diese können zu einem linearen Gleichungssystem umgeformt werden: 
\begin{equation}
    z_{L}= H_{L}\cdot x + v
\end{equation}

Die einzelnen Vektoren und Matrizen lauten:
\setcounter{MaxMatrixCols}{15}
\begin{equation}
	z_{L} = (\Delta\dot \alpha-q \;\; \dot q \;\; \Delta\dot V_A)^T
\end{equation}

\begin{equation}
	 H_{L} = \begin{pmatrix}
		0&0&0& \frac{-\sin{(\alpha_0)}\cdot\Delta\delta_F}{V_0} & \frac{\Delta\alpha}{V_0}& \frac{\Delta V_A}{V_0} & 
		\frac{\Delta\eta}{V_0} &0&0&0&0&0   \\
		0&0&0&0&0&0&0 &\Delta\alpha & q & \Delta V_A & \Delta\eta & \Delta\delta_F \\
		\Delta\alpha &  \Delta V_A & \Delta\eta & \cos{(\alpha_0)}\cdot\Delta\delta_F &0&0&0&0&0&0&0&0 
	\end{pmatrix}
\end{equation}

\begin{equation}
	x = (X_{\alpha}\; 
	X_{\nu}\;
	X_{\eta}\;
	X_{\delta_F}\; 
	Z_{\alpha}\; 
	Z_{\nu}\;
	Z_{\eta}\;
	M_{\alpha}\;
	M_{q}\;
	M_{\nu}\;
	M_{\eta}\;
	M_{\delta_F})^T
\end{equation}

 \todo[inline]{Beschreibung der Vorgehensweise, d.h. Lösung des Gleichungssystems mit LSQ}
 
 \section{Ergebnisse}
 
Anschließend werden die Ergebnisse in diesem Kapitel kurz vorgestellt.

Nach der Schätzung der Einträge der Matrizen $A$ und $B$ wird das Anfangswertproblem \eqref{eq:laengsbewegung} im gesamten 
Zeitintervall gelöst. Bei allen Simulationen sind die Filterparametern wie in \ref{section:filterung} gewählt.

In \cref{fig:Ergebnisse_zmlsq} wird die Lösung anhand des Matrix-LSQ-Verfahrens dargestellt. Die Ergebnisse zeigen eine 
relativ präzise Approximation. Das Verfahren ist sehr schnell und einfach zu implementieren im Vergleich zur 
LSQ-Methode.\todo{Vergleich der Verfahren hier noch nicht so passend}  Allerdings weist das Ergebnis in flugmechanischer 
Hinsicht Abweichungen in System- 
und Eingangsmatrix auf. Die Matrizen $A$ und $B$ lauten: 
\begin{equation}
 	A = \begin{pmatrix}
 		-0.1940 & 0.005 & -0.0002 & -0.0540 \\
 		-38.9899 & -12.8888 & 0.3632 & -4.2686 \\
 		-6.1506 & -0.1501 & -0.0589 & -5.0107 \\
 		0.4433 & 0.0427 & 0.0023 & 0.0818
 	\end{pmatrix}
	\nonumber
\end{equation}

\begin{equation}
 	 B = \begin{pmatrix}
 		-0.0267 & -0.0164 \\ 
 		-26.1377 & -12.9651 \\
 		-0.7414 & 4.6092 \\
 		0.0477 & -0.1328
 	\end{pmatrix}
 \nonumber
\end{equation}

\todo[inline]{Ergebnisse der Matrizen bewerten}
 
\cref{fig:Ergebnisse_zlsq} zeigt die Lösung anhand des LSQ-Verfahrens. Die Methode liefert keine sonderlich guten Ergebnisse. 
\todo{Instabilität des Simulators erwähnen}\todo{warum schlechte Ergebnisse?}
\begin{figure}[h!]
	\centering
	\includegraphics[trim=100 0 100 0,clip,width=1\linewidth]{LS.png}
	\caption{Ergebnisse anhand des Matrix-LSQ-Verfahrens}
    \label{fig:Ergebnisse_zmlsq}
\end{figure}

 \begin{figure}[h!]
	\centering
	\includegraphics[trim=100 0 100 0,clip,width=1\linewidth]{LS_LSQ.png}
	\caption{Ergebnisse anhand des LSQ-Verfahrens}
     \label{fig:Ergebnisse_zlsq} 
\end{figure}

\todo{Diagramme gut lesbar?}
