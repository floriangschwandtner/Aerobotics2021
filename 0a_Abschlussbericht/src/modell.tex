\chapter{Modelle}

Im folgenden Abschnitt werden die der durchgeführten Systemidentifikationen 
zugrundeliegenden Modelle beschrieben, deren beiwerte zu bestimmen sind. Es 
handelt sich dabei um die bekannten linearisierten Modelle der Längs- und 
Seitenbewegung mit den folgenden Annahmen \cite{Fichter2009}:
\begin{itemize}
	\item Linearisierung um den symmetrischen Horizontalflug ($ \gamma_0=0 $)
	\item kein Auftrieb durch Nickrate ($ Z_q=0 $)
	\item keine Querkräfte durch Roll- oder Gierdrehrate ($ Y_p=Y_r=0 $)
	\item keine Querkraft durch Querruder ($ Y_\xi $)
	\item kein Wind ($ \Delta\gamma = \Delta\theta-\Delta\alpha $)
	\item horizontal eingebautes Triebwerk ($ i_F=0 $)
\end{itemize} 
Die Dynamiken können deshalb entkoppelt behandelt werden. %\cite{Vorlesung2}

\section{Längsbewegung}
Der Zustand der Längsbewegung setzt sich zusammen aus dem Anstellwinkel $ 
\alpha $, der Nickrate $ q $, der Anströmgeschwindigkeit $ V_A $ und dem 
Bahnwinkel $ \gamma $. Die zugehörigen Steuerungen umfassen den 
Höhenruderausschlag $ \eta $ und den Schubdrosselgrad $ \delta_F $. Bis auf die 
Nickrate werden alle Größen als Abweichungen (Delta-Größen) vom jeweiligen 
Trimmpunkt 
(gekennzeichnet durch den Index "$ _0 $") beschrieben. Es ergibt sich folgendes 
Modell\cite{Fichter2009}:

\begin{equation}\label{eq:laengsbewegung}
	\begin{pmatrix}
		\dot{\Delta x}\\
		\dot{q}\\
		\dot{\Delta V_A}\\
		\dot{\Delta \gamma}
	\end{pmatrix} = 
	\begin{pmatrix}
		\frac{Z_\alpha}{V_0} & 1 & \frac{Z_V}{V_0} & 0\\
		M_\alpha & M_q & M_V & 0\\
		X_\alpha & 0 & X_V & -g\\
		-\frac{Z_\alpha}{V_0} & 0 & -\frac{Z_V}{V_0} & 0\\
	\end{pmatrix} \cdot
	\begin{pmatrix}
		\Delta x\\
		q\\
		\Delta V_A\\
		\Delta \gamma
	\end{pmatrix} + 
	\begin{pmatrix}
		\frac{Z_\eta}{V_0} & -\frac{X_{\delta F}}{V_0} \sin{(\alpha_0)}\\
		M_\eta & M_{\delta F}\\
		X_\eta & X_{\delta F} \cos{(\alpha_0+i_F)}\\
		-\frac{Z_\eta}{V_0} & \frac{X_{\delta F}}{V_0} \sin{(\alpha_0)}\\
	\end{pmatrix}\cdot
	\begin{pmatrix}
		\Delta \eta\\
		\Delta \delta_F\\
	\end{pmatrix}
\end{equation}

\section{Seitenbewegung}
Das Modell der Seitenbewegung wird mit dem absoluten Zustand aufgestellt:

\begin{equation}\label{eq:seitenbewegung}
	\begin{pmatrix}
		\dot{r}\\
		\dot{\beta}\\
		\dot{p}\\
		\dot{\phi}
	\end{pmatrix} = 
	\begin{pmatrix}
		N_r & N_\beta             & N_p & 0\\
		-1  & \frac{Y_\zeta}{V_0} & 0   & \frac{g}{V_0}\\
		L_r & L_\beta             & L_p & 0\\
		0   & 0                   & 1   & 0\\
	\end{pmatrix} \cdot
	\begin{pmatrix}
		\dot{r}\\
		\dot{\beta}\\
		\dot{p}\\
		\dot{\phi}
	\end{pmatrix} + 
	\begin{pmatrix}
		N_\xi & N_\zeta\\
		0 & \frac{Y_\zeta}{V_0}\\
		L_\xi & L_\zeta\\
		0 & 0\\
	\end{pmatrix}\cdot
	\begin{pmatrix}
		\xi\\
		\zeta
	\end{pmatrix}
\end{equation}

