\chapter{Einleitung}

Im folgenden Abschnitt wird das der durchgeführten Systemidentifikation zugrundeliegende Modell beschrieben. Es handelt sich 
dabei um die bekannten linearisierten Modelle der Längs- und Seitenbewegung 
\cite{Fichter2009} mit den üblichen Annahmen \footnote{Linearisierum den 
symmetrischen Horizontalflug, kein Auftrieb durch Nickrate, keine Querkräfte 
durch Roll- oder Gierdrehrate, keine Querkraft durch Querruder}, deren Beiwerte 
zu bestimmen sind. Die Dynamiken können deshalb entkoppelt behandelt werden. 
%\cite{V2}

\section{Längsbewegung}
Der Zustand der Längsbewegung setzt sich zusammen aus dem Anstellwinkel $ 
\alpha $, der Nickrate $ q $, der Anströmgeschwindigkeit $ V_A $ und dem 
Bahnwinkel $ \gamma $. Die zugehörigen Steuerungen umfassen den 
Höhenruderausschlag $ \eta $ und den Schubdrosselgrad $ \delta_F $. Bis auf die 
Nickrate werden alle GRößen als Abweichungen vom jeweiligen Trimmpunkt 
beschrieben.

\begin{equation}
	\begin{array}{cols}
		\dot{\Delta x}\\
		\dot{q}\\
		\dot{\Delta V_A}\\
		\dot{\Delta \gamma}
	\end{array}
\end{equation}