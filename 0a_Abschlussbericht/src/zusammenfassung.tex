\chapter{Zusammenfassung und Ausblick}
- Datenvorbereitung sehr wichtig und großer Teil der Arbeit\\
- Matrix-LSQ sehr schnell und einfach zu implementieren, ABER flugmechanisch nur bedingt sinnvoll\\
- Frequenzbereich nur sinnvoll bei passendem Eingangssignal\\
\\
- generell: Datennormalisierung (Skalierung mit Maximalwerten => Absolutwerte in gleicher Größenordnung
) bzw. -gewichtung\\
- Zeitbereich: nichtlineare Modellbildung\\
- Frequenzbereich: Equation Error-Methode, kombinierte Methode (Output Error + Equation Error)
