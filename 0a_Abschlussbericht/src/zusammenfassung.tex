\chapter{Zusammenfassung und Ausblick}\label{chapter:Zusammenfassung}
Abschließend lässt sich sagen, dass die Datenvorbereitung ein sehr wichtiger und auch zeitintensiver Teil der gesamten Arbeit 
war. Die richtigen Messsignale aus der Fülle an Originaldaten zu finden und auf Plausibilität zu prüfen legt den Grundstein 
für alle weiteren Schritte. Schwierig ist außerdem die Auswahl eines geeigneten Zeitbereichs zur Identifikation. 

Das LSQ-Verfahren im Zeitbereich lieferte keine sonderlich guten Ergebnisse. Deutlich bessere Approximationen der Messdaten 
ließen sich mit dem Matrix-LSQ-Verfahren erreichen. Dieses stellte sich als sehr schnell und vergleichsweise einfach in der 
Implementierung heraus. Aus flugmechanischer Sicht sind allerdings Einbußen in den bestimmten Parametern hinzunehmen, da sich 
keine Vorgaben bzgl. der bereits bekannten Einträge in der System- und Steuermatrix treffen lassen.

Es zeigte sich, dass der Flug in den Platzrunden weniger geeignet für eine Identifikatition im Frequenzbereich ist. Der 
Verlauf des Steuersignals ist hier von großer Relevanz, bestimmte Manöver, wie ein \textit{Frequency Sweep}, sollten hier zu 
deutlich besseren Ergebissen führen. Nichtsdestotrotz konnten bestimmte Zustandsgrößen angenähert werden. Der 
Implementierungsaufwand für die \textit{Output Error}-Methode ist deutlich höher verglichen mit den Zeitbereich-Verfahren; 
die symbolische Berechnung der sehr länglichen Ableitungen in Matlab ermöglichte überhaupt erst eine sinnvolle Anwendung.\\

An dieser Stelle sei erneut erwähnt, dass in der vorliegenden Arbeit lediglich der Platzrundenflug berücksichtigt wurde. In 
den restlichen Flugdaten steckt aber womöglich ebenso Potenzial zur Bestimmung des flugmechanischen Modells, vor allem, wenn 
sich noch passende Manöver (z.B. Anregung der Eigenschwingung des Flugzeugs) finden lassen. Als nächster Schritt bietet sich 
also eine genaue Untersuchung des bisher vernachlässigten Flugs an.

Möglichkeiten zur Verfeinerung und Optimierung der Identifikation selbst gibt es viele. An erster Stelle bietet sich eine 
Normalisierung der Signale an, d.h. eine Skalierung mit den Maximalwerten eines jeden Verlaufs, sodass die Werte danach in 
der gleichen Größenordnung liegen. Dies ist sonst nicht der Fall, die Geschwindigkeit erreicht weit höhere Absolutwerte als 
alle anderen Zustände, was in den verwendeten Algorithmen einer stärkeren Gewichtung des Geschwindigkeitssignalls entspricht. 
Eine Gewichtung 
ließe sich aber außerdem bewusst vornehmen und damit der Fokus speziell auf wichtige Größen legen. Weiterhin könnten die 
zugrundeliegenden Modelle vereinfacht werden. In der Längsbewegung würde das bedeuten, zwei reduzierte Modelle für die 
Anstellwinkelschwingung einerseits und die Phygoide andererseits zu erstellen und die Identifikationen getrennt durchzuführen.

Weit aufwändiger, aber auch sehr vielversprechend, ist eine Identifikation im Zeitbereich mit einem nichtlinearen 
Modell\todo{Infos ergänzen}.

Im Frequenzbereich wurde mit der \textit{Output Error}-Methode nur der Fehler im gemessenen Signal berücksichtigt. Alternativ 
wäre hier eine Betrachtung des Modellfehlers bei gleichzeitiger Vernachlässigung des Messfehlers möglich, wie es die 
\textit{Equation Error}-Methode vorsieht. Zu guter Letzt lassen sich diese beiden Methoden auch kombinieren.

%- Datenvorbereitung sehr wichtig und großer Teil der Arbeit\\
%- Matrix-LSQ sehr schnell und einfach zu implementieren, ABER flugmechanisch nur bedingt sinnvoll\\
%- Frequenzbereich nur sinnvoll bei passendem Eingangssignal\\
%\\
%- generell: Datennormalisierung (Skalierung mit Maximalwerten => Absolutwerte in gleicher Größenordnung
%) bzw. -gewichtung\\
%- Zeitbereich: nichtlineare Modellbildung\\
%- Frequenzbereich: Equation Error-Methode, kombinierte Methode (Output Error + Equation Error)
%- Vereinfachung des Modells