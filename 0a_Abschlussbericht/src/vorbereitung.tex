\chapter{Vorbereitung der Daten}

Aus den Flugversuchen des e-Genius 1:3 liegt eine Fülle von Messdaten in verschiedenen .csv-Dateien vor. Für die weitere 
Verarbeitung ist es zunächst nötig, aus den gegebenen Messdaten die für die 
Systemidentifikation relevanten auszuwählen bzw. zu berechnen.

\section{Filterung} %Florian

Verrauschte Messdaten stellen für die Systemidentifikation eine Herausforderung dar. Numerische Ableitungen aus verrauschten Daten liefern in vielen Fällen keine sinnvolle Aussage. Neben aufwändigeren Ableitungsregeln bietet sich eine vorangehende Filterung der Daten an.

Das Vorwärts-Rückwärtsfilter bietet den Vorteil, dass keine Phasenverschiebung 
auftritt. Gerade wenn nur einzelne Signalteile gefiltert werden, beispielsweise 
nur der Eingang, ist diese Eigenschaft unerlässlich. Der Nachteil ist, dass das 
Filter nicht in Echtzeit verwendet werden kann, da immer die vollständige 
Datenreihe vorliegen muss. Für eine Systemidentifikation ist dies keine 
praktische Einschränkung.

\subsection{Ablauf}

Für das Filter wird eine Übertragungsfunktion $f(s)$ auf die Messdaten vorwärts 
angewandt, die Messdaten umgekehrt und die selbe Übertragungsfunktion noch 
einmal verwendet. In Matlab ist dies in der Funktion \textit{filtfilt()} 
bereits implementiert. 

\subsection{Wahl der Filterübertragungsfunktion}

Es wurde ein PT2-Glied gewählt, da so die Eckfrequenz direkt eingestellt werden 
kann. % also Tiefpass-Filter?

Mit

\begin{equation}
	\omega_{filt} = 2 \cdot \pi \cdot f_{eck}
\end{equation}

und

\begin{equation}
	\zeta_{filt} = \frac{1}{\sqrt{2}}
\end{equation}

ergibt sich die Übertragungsfunktion zu:

\begin{equation}
	f(s) = \frac{\omega_{filt}^2}{s^2+2 \cdot \zeta_{filt} \cdot \omega_{filt} +\omega_{filt}^2}
\end{equation}