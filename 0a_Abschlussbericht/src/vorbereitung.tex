\chapter{Vorbereitung der Daten}
Aus den Flugversuchen des e-Genius 1:3 liegt eine Fülle von Messdaten in verschiedenen .csv-Dateien vor. Im folgenden Kapitel 
wird erklärt, wie diese vorbereitet werden, um sie für die anschließende Systemidentifikation zu verwenden.



\section{Ermittlung der relevanten Signale} % Fabrizio
%\todo[inline]{Abschnitt "Ermittlung der relevanten Signale" schreiben}
Für die weitere Verarbeitung ist es zunächst nötig, aus den gegebenen Messdaten die relevanten Signalverläufe auszuwählen 
bzw. zu berechnen. Die meisten Zustandsgrößen können direkt aus den Messdaten verwendet werden, einzig der Bahnwinkel $ 
\gamma $ muss explizit berechnet werden. Die Bestimmung über die Beziehung $ \gamma=\theta-\alpha $ liefert dabei aufgrund 
eines unplausiblen Verlaufs des Nickwinkels $ \theta $ kein sinnvolles Ergebnis.\footnote{Berechnet man den Bahnwinkel auf 
diese Weise, nimmt er genau wie der Nickwinkel nie Werte unter 0 an. Lediglich der Anstellwinkel bewegt sich sowohl im 
positiven als auch im negativen Bereich.} Stattdessen wird der Bahnwinkel über \cref{eq:gammaCalculation} berechnet, in 
welcher die Bahngeschwindigkeit mit der Anströmgeschwindigkeit gleichgesetzt wird (kein Wind).

\begin{equation}
	\gamma = \arcsin{\left( \frac{\dot h}{V} \right)} = \arcsin{\left( \frac{\dot h}{V_A} \right)}
	\label{eq:gammaCalculation}
\end{equation}

\todo{Auswahl des zeitlichen Ausschnitts erklären}


\section{Trimmpunkt} % Fabrizio
\todo[inline]{Abschnitt fertig schreiben: Trimmpunkt}
In \cref{fig:trimmpunkte} ist beispielhasft der zeitliche Verlauf der Anströmgeschwindigkeit dargestellt. Es zeigen sich 
insgesamt acht stationären Bereiche, die als Trimmpunkt für eine Modellierung dienen könnten. In den nachfolgenden 
Systemidentifikationen soll jeweils der erste Punkt (TP 1) als Grundlage dienen. Dazu werden alle Zustandsgrößen über den 
Bereich TP 1 gemittelt und diese Mittelwerte als Trimmwerte $ x_0 $ und $ u_0 $ verwendet. Damit lassen sich die Abweichungen 
vom Trimmpunkt berechnen:
\begin{equation}
	\begin{split}
		\Delta x(t) &= x(t)-x_0\\
		\Delta u(t) &= u(t)-u_0
	\end{split}
\end{equation}

Eine Ausnahme bilden hier die Drehraten $ p $, $ q $ und $ r $ , bei denen keine Differenz zum Trimmwert gebildet werden muss.


\begin{figure}
	\centering
	\includegraphics[width=\linewidth]{trimmpunkte.png}
	\caption{zeitlicher Verlauf der Anströmgeschwindigkeit mit den stationären Bereichen}
	\label{fig:trimmpunkte}
\end{figure}




\section{Interpolation} %Florian
\todo[inline]{Abschnitt schreiben: Interpolation}


\section{Filterung} %Florian

Verrauschte Messdaten stellen für die Systemidentifikation eine Herausforderung dar. Numerische Ableitungen aus verrauschten Daten liefern in vielen Fällen keine sinnvolle Aussage. Neben aufwändigeren Ableitungsregeln bietet sich eine vorangehende Filterung der Daten an.

Das Vorwärts-Rückwärtsfilter bietet den Vorteil, dass keine Phasenverschiebung 
auftritt. Gerade wenn nur einzelne Signalteile gefiltert werden, beispielsweise 
nur der Eingang, ist diese Eigenschaft unerlässlich. Der Nachteil ist, dass das 
Filter nicht in Echtzeit verwendet werden kann, da immer die vollständige 
Datenreihe vorliegen muss. Für eine Systemidentifikation ist dies jedoch keine 
praktische Einschränkung.

\subsection{Ablauf}

Für das Filter wird eine Übertragungsfunktion $f(s)$ auf die Messdaten vorwärts 
angewandt, die Messdaten umgekehrt und die selbe Übertragungsfunktion noch 
einmal verwendet. In Matlab ist dies in der Funktion \textit{filtfilt()} 
bereits implementiert. 

\subsection{Wahl der Filterübertragungsfunktion}

Es wurde ein PT2-Glied gewählt, da so die Eckfrequenz direkt eingestellt werden 
kann. % also Tiefpass-Filter?

Mit

\begin{equation}
	\omega_{filt} = 2 \cdot \pi \cdot f_{eck}
\end{equation}

und

\begin{equation}
	\zeta_{filt} = \frac{1}{\sqrt{2}}
\end{equation}

ergibt sich die Übertragungsfunktion zu:

\begin{equation}
	f(s) = \frac{\omega_{filt}^2}{s^2+2 \cdot \zeta_{filt} \cdot \omega_{filt} +\omega_{filt}^2}
\end{equation}