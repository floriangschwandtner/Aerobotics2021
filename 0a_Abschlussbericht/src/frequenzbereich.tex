\chapter{Systemidentifikation im Frequenzbereich}

Neben der Systemidentifikation im Zeitbereich ist es möglich die Parameterschätzung auch im Frequenzbereich durchzuführen. 
Die Analyse im Frequenzbereich bringt einige Vorteile mit sich, die im weiteren Verlauf dieser Dokumentation genauer 
beleuchtet werden. Die Grundlage der Frequenzanalyse ist die Fouriertransformation der gemessenen Daten vom Zeitbereich in 
den Frequenzbereich. Ausgehend von den transformierten Messwerten wird im Folgenden genauer auf die Idee der 
Output-Error-Methode und auf den Algorithmus zur Lösung des Schätzproblems eingegangen. Zum Schluss werden die Ergebnisse 
bewertet und der Blick auf mögliche Verbesserungen und andere Methoden gerichtet.


\section{Fouriertransformation}  

\begin{align}
	j\omega_{k}\mathbf{\tilde{x}}(k)  &= \mathbf{A\tilde{x}}(k) + \mathbf{B\tilde{u}}(k) \nonumber \\
 	\mathbf{\tilde{y}}(k)             &= \mathbf{C\tilde{x}}(k) + \mathbf{D\tilde{u}}(k) = \mathbf{G}(k,\mathbf{\theta})\mathbf{\tilde{u}}(k) \nonumber \\
 	\text{mit }\mathbf{G}(k,\mathbf{\theta}) &= \mathbf{C}(j\omega\mathbf{I}-\mathbf{A})^{-1}\mathbf{B}+\mathbf{D}
	\label{eq:ZRD_Frequenzbereich}
\end{align}

\missingfigure{Bild von Fourier-Trafo}


\section{Output-Error-Methode}

Die Idee der Output-Error-Methode (OEM) liegt in der Art der Fehlerbetrachtung. Es wird dabei der Fehler, der durch die 
mathematische Modellbildung entsteht, vernachlässigt. Das dynamische System wird als deterministisch angenommen. 
Unsicherheiten entstehen nur durch fehlerhafte und verrauschte Messungen. Ausgangspunkt der OEM ist das transformierte 
Zustandsraummodell (\ref{eq:ZRD_Frequenzbereich}). Im vorliegenden Fall entspricht der Ausgangsvektor $\tilde{y}$ dem 
Zustandsvektor $\tilde{x}$. Die Matrizen $\mathbf{C}$ und $\mathbf{D}$ nehmen also folgende Gestalt an:

\begin{align}
	\mathbf{C} &= \mathbf{I} \nonumber \\
 	\mathbf{D} &= \mathbf{0}          
	\label{eq:CD}
\end{align}

Da der Zustandsvektor gleichzeitig der gemessene Zustand ist, kann der Output-Error folgendermaßen formuliert werden:

\begin{equation}
    \mathbf{\tilde{\nu}}(k,\mathbf{\theta}) = \mathbf{\tilde{z}}(k)-\mathbf{\tilde{y}}(k,\mathbf{\theta}) = \mathbf{\tilde{z}}(k)-\mathbf{G}(k,\mathbf{\theta})\mathbf{\tilde{u}}(k)  
	\label{eq:Output_Error}
\end{equation}

$\mathbf{\tilde{z}}$ entspricht dabei dem gemessenen Zustand, der durch Messrauschen aus dem übertragenen Zustand 
$\mathbf{\tilde{y}}$ entsteht. Der übertragene Zustand kann aus den Steuergrößen $\mathbf{\tilde{u}}$ durch Multiplikation 
mit der Übertragungsmatrix $\mathbf{G}(\mathbf{\theta)}$ berechnet werden. Die Übertagungsmatrix ist, wie in 
\cref{eq:ZRD_Frequenzbereich} gezeigt, abhängig vom zugrunde liegenden Modell und von den zu schätzenden Parametern 
$\mathbf{\theta}$. Diese sind die Einträge der Matrizen des Zustandsraummodells. Das Ziel der Methode ist nun den Ausgangsvektor, der aus dem Modell und seinen Parametern folgt, dem gemessenen Zustand möglichst gut anzunähern. Mathematisch bedeutet das, dass das Minimum einer Kostenfunktion gefunden werden soll. Die Kostenfunktion der OEM ist nach Klein und Morelli \cite{Klein2006} die negative Log-likelihood Funktion mithilfe des Ansatzes des Ausgangsfehlers (Gl. \ref{eq:Output_Error}):

 \begin{equation}
    J(\mathbf{\theta})=N \sum\limits_{k=0}^{N-1}\mathbf{\tilde{\nu}}^H(k,\mathbf{\theta})\mathbf{S_{\nu\nu}^{-1}}\mathbf{\tilde{\nu}}(k,\mathbf{\theta})+Nln|\mathbf{S_{\nu\nu}}|
	\label{eq:Kostenfunktion}
\end{equation}  

\todo{Svv kurz in ein zwei Sätzen erklären}
Wie bereits beschrieben geht es nun darum das Minimum dieser Kostenfunktion zu finden. Wir suchen also die Nullstelle ihrer Ableitung. Einer der wichtigsten Algorithmen zur Bestimmung von Nullstellen sowohl bei Eingrößenproblemen, als auch bei Mehrgrößenproblemen ist der iterative Newton Raphson Algorithmus.

\section{Ergebnisse}