\chapter{Einleitung}
Die vorliegende Arbeit wurde im Rahmen des Aerobotics-Seminars im Sommersemester 2021 an der Universität Stuttgart verfasst. 
Ziel des Projekts war die Identifikation der Parameter eines linearisierten Modells der Längsbewegung aus vorliegenden 
Flugmessdaten eines 1:3-Modells des Flugzeugs \textit{e-Genius}. Für die Seitenbewegung wurde ebenfalls eine 
Systemidentifikation durchgeführt, aus Gründen der Übersichtlichkeit und Kürze wird diese aber in der Ausarbeitung 
ausgelassen.\par
Es wurden Verfahren im Zeit- und im Frequenzbereich verwendet, um die Ergebnisse abschließend vergleichen zu können. Genauer 
handelt es sich dabei im Zeitbereich um die \textit{Least Squares}-Methode sowie die Matrix-LSQ-Methode. Im Frequenzbereich 
wurde die \textit{Output Error}-Methode in Verbindung mit einem \textit{Newton-Raphson}-Algorithmus verwendet.\\

Linearisierte Modelle spielen eine wichtige Rolle in der Regelungstechnik: Mit ihnen lässt sich eine lineare Regelung um 
einen stationären Arbeits- bzw. Trimmpunkt entwickeln. Eine Schwierigkeit besteht dabei aber immer wieder darin, die 
Regelstrecke mit dem Modell ausreichend genau zu beschreiben. Hier kommt die Systemidentifikation ins Spiel: Sie hat zum 
Ziel, anhand gegebener Flugmessdaten die Parameter des definierten Modells bestmöglich abzuschätzen.